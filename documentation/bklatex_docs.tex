\documentclass[a4paper, 10pt]{article}

\usepackage{geometry}
	\geometry{margin = 1in}
\usepackage{listings}
\usepackage{lstautogobble}
\usepackage{xcolor}


\lstset{
	language = Python,
	tabsize = 4,
	basicstyle = \ttfamily\small,
	breaklines = true,
	autogobble = true,
	numbers = left,
	numberstyle = \tiny\color{black!70},
	commentstyle = \color{black!70},
	keywordstyle = ,
	xrightmargin = .1\linewidth,
	aboveskip = .5\baselineskip,
	belowskip = \baselineskip,
	showstringspaces = false
}

\newcommand{\bklatex}{{\ttfamily bklatex}}


\title{{\ttfamily \bklatex}: Bookkeeping with {\LaTeX}}
\author{Naman Taggar\footnote{{\ttfamily namantaggar.11@gmail.co}m}}
\date{01 January, 2024}

\begin{document}
	\maketitle
	
	\tableofcontents
	
	\section{Introduction}
	\label{sec:intro}
	
	{\bklatex} is a Python library inspired from {\ttfamily pylatex} made by Jelte Fennema, as he describes it:
	\begin{center}
		\parbox{.8\linewidth}{
			{\itshape PyLaTeX is a Python library for creating and compiling {\LaTeX} files or snippets. The goal of this library is being an easy, but extensible interface between Python and {\LaTeX}.}
		}
	\end{center}
	{\bklatex} is used to typeset accounting journals and ledgers easily using Python. Python modules are used to constitute entries which are stored in appropriate databases, to then be iterated through for printing {\LaTeX} commands using pre-defined methods.
	
	\section{Complete Reference}
	\begin{itemize}
		\item {\ttfamily account} class: stores database in dictionary form
		\begin{itemize}
			\item {\ttfamily month} class: stores entries for each month in list form
			\begin{itemize}
				\item {\ttfamily what}
			\end{itemize}
		\end{itemize}
	\end{itemize}
	
	
	\section{Source-code}
	\lstset{firstnumber = last} W.I.P.
\end{document}